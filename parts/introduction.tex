% INTRODUCTION - BY Jonas Frede
% !TEX root = ../MathLog.tex

\chapter*{\IfLanguageName{english}{Preface}{Vorwort}}

\IfLanguageName{english}{\todonote{translate here}}{Im Internet stieß ich auf einen Beitrag einer Person, welche ein Logbuch führt, in dem sie mathematische Fakten, Ideen und Gelerntes festhält. Daraus erwuchs für mich das Bedürfnis, neben neu erlerntem Stoff auch gelernte Dinge zu wiederholen, indem ich alte Mitschriften meiner Vorlesungen digitalisiere und mit eigenen Kommentaren erweitere.

Dieses Dokument soll dem oben genannten Zweck dienen, und wird daher Gegenstand ständiger Veränderung und Erweiterung sein. Das Datum der letzten Änderung findet sich auf der Titelseite.

Eventuell können Teile dieses Projektes auch für andere Menschen von Nutzen sein. Daher wird oft in einem einladenden und in der Mathematik üblichen Ton von \enquote{wir} gesprochen. Ich freue mich über Hinweise auf Fehler und Tipps, da ich keinesfalls vorgeben möchte, in allen vorgestellten Bereichen ein Experte zu sein. Viel Spaß beim Lesen!}

\mbox{}\newline
  \hspace*{\fill} \IfLanguageName{english}{Berlin, January 25th, 2017}{Berlin, den 25.01.2017} \\
  \hspace*{\fill} Jonas Frede

\newpage
\section*{\IfLanguageName{english}{Recommendation for the reader}{Empfehlung für den Leser}}
\IfLanguageName{english}{\todonote{translate here}}{Natürlich kann man als interessierter Leser wie Sie dort einsteigen, wo man möchte. Jemandem, der ein wenig Anleitung haben möchte, empfehle ich folgendes Diagramm, in welcher Reihenfolge eine Lektüre empfehlenswert ist:}

\todowarning{diagram for order of chapters to read}

\IfLanguageName{english}{\todonote{translate here}}{Der unerfahrene und lernende Leser sei dazu angehalten, die im Text getätigten Aussagen zu verifizieren. Der Inhalt mag anfangs einfach und offensichtlich wirken, allerdings ist empfohlen, die Art und Form der Wissensvermittlung zu berücksichtigen, da diese besonders wichtig ist und der Mathematik eigen ist: Aussagen (wie ein \emph{Satz, Lemma} oder ähnliches) müssen grundsätzlich auf bereits bekanntes Wissen und vorher bekannte Aussagen logisch zurückgeführt werden, mit Hilfe eines \emph{Beweises}.
Man sollte daher versuchen, jede Aussage und jeden Beweisschritt nachzuvollziehen und, falls möglich, mitzuschreiben. Nicht zuletzt, um die Lesegeschwindigkeit zu drosseln. Fügen Sie Details, die ihrer Meinung nach im Text fehlen, ein, und teilen Sie mir gerne Verbesserungsvorschläge mit.

Fragen, die beim Lesen des Textes an verschiedenen Stellen helfen sollen, sind beispielsweise \enquote{Kann ich die Definition oder den Satz in eigenen Worten wiedergeben?}, \enquote{Kenne ich ein Beispiel für die beschriebene Definition oder Voraussetzung?} und \enquote{Kenne ich ein Beispiel, für das die Voraussetzungen einer Aussage nicht erfüllt sind?}.
Zu Sätzen sollte man sich zusätzlich fragen \enquote{Was sind die Voraussetzungen, was ist die Behauptung?}, und bevor man einen Beweis beginnt zu lesen, ist es hilfreich, sich zuerst selbst die Frage zu beantworten \enquote{Was ist in einem Beweis zu tun, um aus den Voraussetzungen die Behauptung herzuleiten?}.
In der Erarbeitung eines Beweises der Form \enquote{Aus \(A\) folgt \(B\)} ist es schließlich wichtig, sich zu fragen \enquote{Habe ich verstanden, warum \(B\) aus \(A\) folgt?}.

Für eine etwas ausführlichere Einführung in das mathematische Arbeiten sei auf das lehrreiche Skript und Buch~\cite{SchicS:Einfuehrungindas} und dessen Quellen verwiesen.}
